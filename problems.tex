\documentclass[addpoints,12pt,a4paper]{exam}

\printanswers

\makeatletter
\expandafter\providecommand\expandafter*\csname ver@framed.sty\endcsname
{2003/07/21 v0.8a Simulated by exam}
\makeatother

\usepackage{xcolor}
\usepackage{minted}
\usepackage[utf8]{inputenc}
\usepackage{tikz}
\usepackage{caption}
\usepackage{gensymb}
\usepackage{lmodern}
\usepackage{multirow}
\usepackage{booktabs}
\usepackage{array}
\usepackage{adjustbox}
\usepackage{upquote}
\usepackage{amsmath}
\usepackage[hidelinks]{hyperref}
\usepackage{fontawesome}

\usetikzlibrary{}

\newcolumntype{x}[1]{>{\centering\arraybackslash\hspace{0pt}}p{#1}}

\pagestyle{headandfoot}
\header{\textbf{Problem Sheet: Racket}}{}{\href{https://github.com/theory-of-algorithms}{Theory of Algorithms \faGithub}}
\footer{}{Page \thepage\ of \numpages}{}


\renewcommand{\refname}{\selectfont\normalsize References} 


\begin{coverpages}
\end{coverpages}

\begin{document}



\noindent
The following exercises are related to the Racket programming language~\cite{racketwebsite}.

\begin{questions}

\question
Re-write the following expressions in Racket and evaluate them using a Racket interpreter/compiler.
\begin{parts}
  \part $(3 \times (5 + (10 \div 5)))$
  \part $(2 + 3 + 4 + 5)$
  \part $(1 + (5 + (2 + (10 \div 3))))$
  \part $(1 + (5 + (2 + (10 \div 3.0))))$
  \part $(3 + 5) \times (10 \div 2)$
  \part $(3 + 5) \times (10 \div 2) + (1 + (5 + (2 + (10 \div 3))))$
\end{parts}


\begin{solution}
\begin{parts}
  \part \mintinline{scheme}{(* 3 (+ 5 (/ 10 5)))}
  \part \mintinline{scheme}{(+ 2 3 4 5)}
  \part \mintinline{scheme}{(+ 1 (+ 5 (+ 2 (/ 10 3))))}
  \part \mintinline{scheme}{(+ 1 (+ 5 (+ 2 (/ 10 3.0))))}
  \part \mintinline{scheme}{(* (+ 3 5) (/ 10 2))}
  \part \mintinline{scheme}{(+ (* (+ 3 5) (/ 10 2)) (+ 1 (+ 5 (+ 2 (/ 10 3))))))}
\end{parts}
\end{solution}


\question
Define a procedure \mintinline{scheme}{discount} that takes two arguments: an item’s initial price and a percentage discount \cite{simplyscheme}.
It should return the new price:
\begin{minted}{scheme}
> (discount 10 5)
9.50
> (discount 29.90 50)
14.95
\end{minted}


\begin{solution}
  \begin{minted}{scheme}
(define (discount p d)
  (* p (- 1 (/ d 100.0))))
  \end{minted}
\end{solution}


\question
Define a function \mintinline{scheme}{grcomdiv} that takes two integers and returns their greatest common divisor.
\begin{minted}{scheme}
> (grcomdiv 10 15)
5
> (grcomdiv 64 30)
2
\end{minted}


\begin{solution}
  \begin{minted}{scheme}
(define (grcomdiv a b)
  (if (< a b)
    (grcomdiv b a)
    (if (= 0 b)
      a
      (grcomdiv b (modulo a b)))))
  \end{minted}
\end{solution}

\question
Write a function called \mintinline{scheme}{appearances} that returns the number of times its first argument appears as a member of its second argument~\cite{simplyscheme}.

\begin{solution}
  \begin{minted}{scheme}
(define (appearances i l)
  (if (null? l)
    0
    (if (equal? i (car l))  
      (+ 1 (appearances i (cdr l)))
      (appearances i (cdr l)))))   
  \end{minted}
\end{solution}

\question
Write a procedure \mintinline{scheme}{inter} that takes two lists as arguments.
It should return a list containing every element that appears in both lists, exactly once.

\begin{solution}
  \begin{minted}{scheme}
; Use appearances from previous question.
(define (inter l1 l2)
  (if (null? l1)
    '()
    (if (> (appearances (car l1) l2) 0)
      (cons (car l1) (inter (cdr l1) l2))
      (inter (cdr l1) l2))))
  \end{minted}
\end{solution}

\question
Write a procedure \mintinline{scheme}{noatoms} that takes a list and returns the number of atoms it contains.

\begin{solution}
  \begin{minted}{scheme}
(define (noatoms l)
  (if (null? l)
    0
    (if (not (or (pair? (car l)) (null? (car l))))
      (+ 1 (noatoms (cdr l)))
      (noatoms (cdr l)))))
  \end{minted}
\end{solution}

\question
Here is a Racket procedure that never finishes its job when $n$ is not $0$:
\begin{minted}{scheme}
(define (forever n)
  (if (= n 0)
       1
       (+ 1 (forever n))))
\end{minted}
Explain why it doesn’t give any result\cite{simplyscheme}.

\begin{solution}
The terminating condition is: $n$ equals 0.
However, each time forever is called, $n$ is increased.
\end{solution}

\question
Write a function called \mintinline{scheme}{range} that takes an integer $n$ and returns a list containing the atoms 1, 2, 3, \ldots, $n$.

\begin{solution}
\begin{minted}{scheme}
(define (range n)
  (if (= n 0)
    '()
    (append (range (- n 1)) (list n))))
\end{minted}
\end{solution}

\question
Write a function called \mintinline{scheme}{reversel} that takes a list and returns it reversed.  

\begin{solution}
\begin{minted}{scheme}
(define (reversel l)
  (define (reversel-aux l a)
    (if (null? l)
      a
      (reversel-aux (cdr l) (cons (car l) a))))
  (reversel-aux l null))
\end{minted}
\end{solution}

\question
If we list all the natural numbers below 10 that are multiples of 3 or 5, we get 3, 5, 6 and 9.
The sum of these multiples is 23.
Write a procedure to find the sum of all the multiples of 3 or 5 below 1000~\cite{projecteuler}.

\begin{solution}
  \begin{minted}{scheme}
(define 
  (sum35 n)
    (if (= 0 n)
      0
      (if (= 0 (modulo n 3))
      (+ n (sum35 (- n 1)))
      (if (= 0 (modulo n 5))
      (+ n (sum35 (- n 1)))
      (sum35 (- n 1))))))

; Call this with 999 for < 1000.
(sum35 999)
  \end{minted}
\end{solution}

\question
Write a procedure called \mintinline{scheme}{flatten} that takes as its argument a list, possibly including sublists, but whose ultimate building blocks are atoms.
It should return a sentence containing all the atoms of the list, in the order in which they appear in the original:
\begin{minted}{scheme}
> (flatten '(((a b) c (d e)) (f g) ((((h))) (i j) k)))
(a b c d e f g h i j k)
\end{minted}

\begin{solution}
  \begin{minted}{scheme}
(define (flatten l)
  (if (null? l)
    '()
    (if (pair? (car l))
      (append (flatten (car l)) (flatten (cdr l)))
      (cons (car l) (flatten (cdr l))))))
  \end{minted}
\end{solution}

\question
Each new term in the Fibonacci sequence is generated by adding the previous two terms.
By starting with 1 and 2, the first 10 terms will be:
\[ 1, 2, 3, 5, 8, 13, 21, 34, 55, 89 \]
By considering the terms in the Fibonacci sequence whose values do not exceed four million, find the sum of the even-valued terms~\cite{projecteuler}.

\begin{solution}
  \begin{minted}{scheme}
(define (sumflt total)
  (define (aux n-2 n-1 maxval total)
    (if (> (+ n-2 n-1) maxval)
      total
      (if (= 0 (modulo (+ n-2 n-1) 2))
        (aux n-1 (+ n-2 n-1) maxval (+ total n-2 n-1))
        (aux n-1 (+ n-2 n-1) maxval total))))
  (aux 0 1 total 0))
  \end{minted}
\end{solution}

\question
Write a procedure \mintinline{scheme}{to-binary} that takes a decimal interger and converts it into a list of 0's and 1's representing the number in binary form.
The least significant bit should be on the right of the list.
\begin{minted}{scheme}
> (to-binary 9)
1001
> (to-binary 23)
10111
\end{minted}

\begin{solution}
  \begin{minted}{scheme}
(define (to-binary n) (if (= n 0)
  '() (append
    (to-binary (/ (- n (modulo n 2)) 2))
    (list (modulo n 2)))))
  \end{minted}
\end{solution}

\question
Write a function \mintinline{scheme}{select} that takes two elements, a list and a position in the list, and return the element of the list in that position.
\begin{minted}{scheme}
> (select (list 1 2 3 4 5) 1)
2
\end{minted}

\begin{solution}
  \begin{minted}{scheme}
(define (select l i)
  (if (= 0 i)
      (car l)
      (select (cdr l) (- i 1))))
  \end{minted}
\end{solution}

\question
Write a function \mintinline{scheme}{perms} that takes a list as its only argument, and returns a list containing all permutations of that list.

\begin{solution}
\begin{minted}{scheme}

(define (perms l)
  (if (null? l)
    '(())
    (apply append
      (map
       (lambda (i) (map
         (lambda (j)(cons i j))
         (perms (remove i l))))
        l))))
  \end{minted}

See the following URL for more information:

\url{https://see.stanford.edu/materials/icsppcs107/32-Scheme-Examples.pdf}
\end{solution}

\end{questions}

\bibliographystyle{plain}
\bibliography{bibliography}
\end{document}
